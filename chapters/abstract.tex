\begin{abstract}
Crohn’s disease is a severe inflammatory bowel disease without a cure. Early diagnosis via T2-weighted MRI scanning of the terminal ileum can significantly improve patient outcomes. To facilitate this, we developed an enhanced segmentation model based on prior work to improve diagnostic accuracy. We used the SegmentAnything Model (SAM) from Meta AI and adapted it to a two-stage training process due to limited human-annotated segmentations. In the first stage, we employed a MedSAM variant to create a refined weak mask from bounding box information derived from centerline coordinates. This enabled us to build a proxy model using nnU-Net in 2D and 3D full-resolution configurations. The second stage involved training this model on fully annotated data. A rigorous 5-fold cross-validation process validated our model. We further identified the best-performing model using an ensemble method across various configurations and folds. Our approach showed superior performance in weak label generation and segmentation compared to previous studies, leading to improved Dice Similarity Coefficient metrics. Our results confirm the potential of pre-trained models and suggest that higher-quality weak labels could significantly enhance segmentation outcomes. This study brings us closer to more effective Crohn’s disease management strategies, indicating promising results with small datasets and nuanced multi-class segmentation differentiating between normal and abnormal terminal ileum tissues.
\end{abstract}