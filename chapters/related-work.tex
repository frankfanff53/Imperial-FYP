\section{Automatic Detection and Segmentation of Crohn's Disease Tissues from Abdominal MRI}

The origins of applying deep learning techniques to medical image analysis can be traced back to approximately a decade ago. In 2013, Mahapara et al. \cite{mahapatra2013automatic} pioneered a machine learning-based method for segmenting bowel regions to detect Crohn's disease tissues in MRI scans.

The proposed pipeline begins with over-segmenting the input MR image test volume into supervoxels. Random Forest (RF) classifiers identify supervoxels containing diseased tissues, subsequently defining the Volume of Interest (VOI). Within the VOI, voxels are further examined to segment the affected region. An additional set of RF classifiers is applied to the test volume to generate a probability map, delineating the likelihood of each voxel being classified as diseased tissue, normal tissue, or background.

Empirical results demonstrate that this approach achieved satisfactory segmentation performance, evidenced by a Dice metric value of \(0.90 \pm 0.04\) and a Hausdorff distance of \(7.3 \pm 0.8\) mm. The significance of this research lies in developing an automated pipeline for segmenting diseased bowel sections in abdominal MR images. This pipeline assists medical experts in identifying affected tissues, thereby facilitating the diagnosis and treatment of Crohn's Disease. The clinical validation of the results, showcasing high segmentation accuracy, further underscores its utility in supporting medical professionals in their work.

Nevertheless, the method is limited by computational inefficiency and complexity due to extended testing times for each instance, fine-tuning requirements for each pipeline stage, and ample opportunities for architectural improvements. Moreover, the research does not delve into finer details, such as the terminal ileum, which is crucial for comprehensive analysis and early diagnosis.


\section{Automatic Detection of Bowel Disease with Residual Networks}

Building on several years of research in the field, Holland et al. \cite{holland2019automatic} put forth a pioneering approach in 2019 to automate the detection of Crohn's disease from a limited dataset of MRI scans. The authors employed an end-to-end residual network \cite{he2016deep}, equipped by a soft attention layer \cite{schlemper2019attention}. This layer essentially magnified salient local features and added a layer of interpretability, providing a clearer understanding of the analytical process.

Their approach exclusively targets the terminal ileum in a strategic departure from semantic segmentation strategies typically employed. This focus underscored the potential feasibility of deep learning algorithms for precisely identifying terminal ileum Crohn's Disease within abdominal MRI scans.

The method's robustness is reflected in its experimental results. Under conditions of localised data within a semi-automatic setting, the model achieved a commendable weighted-f1 score of 0.83. This score is particularly noteworthy given its close correlation with the MaRIA \cite{rimola2009magnetic} score, a clinical standard that enjoys widespread acceptance in the medical community. Beyond its performance metrics, the researchers accentuated the relative efficiency of their model, which necessitated only a fraction of the preparation and inference time compared to standard procedures. This aspect underlines the potential for significant time-saving benefits in a clinical context.

However, the research did reveal certain limitations. Notably, when applied in a fully automatic setting, the model's performance exhibited a marginal decrease in efficiency. Although this does not detract from the study's overall achievements, it highlights an area where further refinement and improvement could be pursued.

Reiterating their discoveries, Holland et al. proposed a strong correlation between model performance and the degree of localisation in the training data. They suggested the collection of gold-standard segmentation of the terminal ileum could prove beneficial as an antecedent task in enhancing automatic detection performance. This proposition opens up intriguing possibilities for research, including the work presented in this thesis, which explores these aspects in greater detail.

Their insights illuminate the synergistic potential between manual analysis and automated methods in enhancing diagnostic capabilities. Importantly, they establish a pathway for integrating deep learning techniques to detect Crohn's disease from limited datasets, indicating a promising approach to tackle one of the significant challenges in machine learning: data scarcity. By leveraging soft attention mechanisms to intensify salient local features and augment interpretability, they provide a valuable tool for medical professionals to comprehend better the results generated by the algorithm.

These findings, particularly the proposed use of gold-standard segmentation of the terminal ileum, provide a solid foundation for the work pursued in this thesis.

\section{Leveraging Machine Learning Methods for Accurate Prediction of Intestinal Damage in Crohn's Disease Patients}

In 2020, Enchakalody et al. \cite{enchakalody2020machine} embarked on an innovative study exploring the potential of machine learning methodologies to enhance the precision and reliability of diagnosing and monitoring Crohn's Disease. They applied these techniques to a small dataset of 207 CT-Enterography (CTE) scans, an approach that mirrors our research focus. Their comprehensive analysis involved examining cross-sectional views of small intestine segments and precisely detecting diseased tissues. Utilising two distinct classifier types - Random Forest (RF) with ensemble techniques and Convolutional Neural Network (CNN) algorithms, they quantitatively evaluated intestinal damage related to Crohn's Disease on each mini-segments.

The efficacy of both RF and CNN techniques was compellingly demonstrated in the experimental results, achieving accuracy rates of 96.3\% and 90.7\%, respectively, for classifying diseased and normal segments. Remarkably, these techniques mirrored the effectiveness of expert radiologists in distinguishing between diseased and normal small bowel tissue. This underscores the immense potential of machine learning, even when applied to small datasets, in elevating the precision of Crohn's Disease diagnoses.

The research conducted by Enchakalody et al. is particularly insightful for our work. It demonstrates the successful application of deep learning techniques on small datasets and opens the door to potentially revolutionising the diagnosis and treatment of Crohn's disease through machine learning. It highlights the possibility of a more precise and automated approach to detecting intestinal damage in such patients, a focus that aligns closely with our current research aims.

While it should be noted that this study primarily focused on data derived from CT-Enterography, differing slightly from our focus on MRI data, the methodology and findings offer valuable insights. As of the time of writing this report, despite the progress made, achieving a fully automatic approach for diagnosing Crohn's disease based on cross-sectional imaging that equates to the proficiency of expert radiologists continues to be an exciting area of ongoing research.