Given the sensitive nature of our project that involves the handling of medical data, we are steadfastly committed to ensuring a robust ethical framework guide all phases of our work. This initiative encompasses further processing and augmentation of previously collected medical data and merging existing datasets.

To safeguard personal identity and ensure strict adherence to privacy standards, all MRI scans are carefully processed under the oversight of clinical radiologists at St Mark Hospital. Comprehensive measures remove any personally identifiable data from the medical records, such as names, genders, ages, and ID numbers. As a result, it is impossible to trace any individual's identity from the processed medical data.

Our commitment to ethical considerations extends into the model training procedure. We employ the nnU-Net framework, which utilises convolutional layers to learn from data via feature extraction. This learning procedure does not store original training data; instead, it creates feature maps representing distilled, valuable insights from the data. This process ensures that the original training data cannot be recovered from the model, thereby preserving individual privacy.

Our project is on a foundational ethical commitment that respects personal anonymity and data confidentiality. Recognising the crucial importance of trust in scientific inquiry, mainly when dealing with sensitive medical data, we are dedicated to exemplifying conscientious practices that uphold the highest standards of research ethics.