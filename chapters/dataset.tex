
The focus of this thesis is the segmentation of medical images, specifically Magnetic Resonance (MR) images, with manually annotated gold standard segmentations of the colon and terminal ileum. MR imaging offers a rich depth of detail, attributable to its three-dimensional nature. This complexity, while advantageous for diagnosis and treatment planning, presents a unique set of challenges for image segmentation. It is imminently pertinent that the orientation in which these images are captured significantly impacts their interpretability and subsequent processing. In this chapter, we will delve into how our imaging data are obtained and perform exploratory analysis on the dataset.

\section{Data Acquisition and Classification}

At the heart of our research lies the utilisation of T2-weighted images, a type of MRI scan that provides detailed pictures of the inside of the body, to help with segmentation tasks. T2-weighted images are a specific genre of magnetic resonance imaging (MRI) scans manifested by the heightened intensity in fluid-rich structures, while their fat-laden counterparts appear darker. This distinctive contrast owes its origin to the heterogeneous responses of different tissues to the magnetic field and radio waves deployed during the MRI procedure.

The unparalleled detailing offered by T2-weighted images of internal human structures, especially soft tissues, makes them an invaluable asset in visualizing bodily fluids, identifying edema or swelling, and surfacing lesions - a potential indicator of Crohn's disease if detected in the gastrointestinal tract. Furthermore, they have been instrumental in diagnosing and assessing critical diseases, such as cancer and multiple sclerosis.

In enriching the diversity of our dataset, we have judiciously incorporated three variants of MR images. A summary describing the unique features and applications of each variant is as follows:

\begin{itemize}
    \item \textbf{Axial T2-weighted Images}: Projected along the axial plane, these images provide a top-down representation of anatomical structures and are especially beneficial in examining anatomical correlations.
    \item \textbf{Coronal T2-weighted Images}: Imaged along the coronal plane, these scans offer a frontal perspective of the anatomy. These images are reputable for their proficient highlighting of fluid-filled structures and lesions.
    \item \textbf{Post-Contrast Axial T2 Images}: These images are captured after administering a contrast agent and enhance tissue contrast, thereby providing enriched insights into the nature of lesions.
\end{itemize}


The strategic incorporation of three distinct forms of MR images enriches our dataset, transforming it into a comprehensive repository designed to master the intricate task of segmentation. Yet, it is not devoid of potential obstacles such as MRI artefacts originating from patient movement or the overarching issue of data scarcity.

The limited availability of annotated data poses dual challenges—it affects not only the trajectory of the model training but also curtails our ability to effectively evaluate the model performance due to the restricted availability of testing data.

However, it is precisely this scarcity of data that fuels our quest for alternative annotated datasets to augment model training. The following chapter provides a detailed discussion of our innovative approach to addressing this challenge.

\section{Dataset Specification}

Our research utilises a carefully compiled dataset comprising 233 MR Images per class. This dataset balances representation with 113 abnormal cases and 120 normal cases for each image type. In addition to images, we also have a collection of centerline coordinates representing the colon in the MR image, although these are not available for every image. Furthermore, a select set of human-annotated ground truth results is included.

Upon closer examination, the centerline and ground truth annotation distribution is as follows:

\begin{table}[htp]
    \centering
    \begin{tabular}{|c|c|c|c|}
    \hline
    Type of Image & Total Centerlines & \begin{tabular}[c]{@{}c@{}}Centerlines\\ (abnormal:normal)\end{tabular} & \begin{tabular}[c]{@{}c@{}}Ground Truth\\ (abnormal:normal)\end{tabular} \\
    \hline
    Axial T2 & 103 & 59:44 & 18:20 \\
    \hline
    Coronal T2 & 93 & 46:47 & 18:30 \\
    \hline
    \begin{tabular}[c]{@{}c@{}}Post-Contrast\\ Axial T2\end{tabular} & - & - & 13:20 \\
    \hline
    \end{tabular}
    \caption{Distribution of centerlines and ground truth segmentations for different MR images}
    \label{table:dataset}
    \end{table}

    The generation of colon centerline coordinates is a meticulous process, performed by radiologists via manual slice-by-slice inspection. Following their careful examination, they annotate the relevant slices with reference points linked to the colon. These annotations are archived as compressed XML files, colloquially referred to as \textbf{traces files}.

    This storage format serves a dual purpose. Firstly, it facilitates visual inspection and analysis within the framework of medical imaging by clinical experts. Secondly, it empowers developers to navigate through the XML tree to gather valuable information about the centerline coordinates. It is worth noting that the preliminary 20\% of the points are often deemed the most accurate, typically representing the interval in which the terminal ileum is situated.

Our diverse and comprehensive dataset provides thoughtful insights for analysis and a solid foundation for reliable model training.

\section{Ground Truth Segmentations}

The ground truth segmentations delivered by our panel of clinical experts are assigned with varying semantic meanings, each corresponding to unique label IDs. The specifics of these relationships are highlighted in Table \ref{tab:segmentation-label-id}. To gain a clear overview of the segmentation, I have extracted a subset of ten samples from each type of image data and performed manual segmentation. The ensuing label distribution, represented on a logarithmic scale, is graphically demonstrated in Figure \autoref{fig:label-distribution}.

\begin{figure}[htp]
    \centering
    \begin{subtable}[htp]{0.48\textwidth}
        \centering
        \begin{tabular}{|c|c|}
            \hline
            Label & ID \\
            \hline
            Background & 0 \\
            \hline
            Abnormal T.I. & 1 \\
            \hline
            Normal T.I. & 2 \\
            \hline
            Colon & 3 \\
            \hline
            Colon & 4 \\
            \hline
            Appendix & 6 \\
            \hline 
        \end{tabular}
        \caption{Label-ID representations used across manual segmentations}
        \label{tab:segmentation-label-id}
    \end{subtable}
    \hfill
    \begin{subfigure}[htp]{0.5\textwidth}
        \centering
        \includegraphics[width=\textwidth]{./figures/label_distribution.png}
        \caption{Illustration of Segmentation Label Distribution}
        \label{fig:label-distribution}
    \end{subfigure}
       \caption{Detailed breakdown of segmentation specifics and label distribution across diverse categories of image data}
       \label{fig:manual-segmentation}
\end{figure}

From the results, a salient observation is the predominance of the background label in the voxel classes of the ground truth segmentation results. Its pronounced presence approximates to one, rendering other voxel labels relatively insignificant and emphasizing the challenges intrinsic to abdominal MRI segmentation.

This raised a need for us to rethink the segmentation scope. Given the relative insignificance of portions other than the background, the informational value they contribute toward the training process is marginal at best. This limited contribution not only impedes the ability of model to classify effectively within these regions but increases the risk of false-positive classifications considering the comparison between the remaining regions versus the background. However, to alleviate this issue, we simplify the scenario into a binary classification problem, interpreting all non-background voxels as general T.I. regions. Accurate segmentation of this region holds significant implications for enabling clinical experts to diagnose Crohn's disease at an early stage.

This insight has inspired us to consider utilizing the centerline coordinates as a guiding factor for future initiatives. Specifically, we are contemplating two potential applications: augmenting the data through bounding box applications or minimizing the search space by cropping the image. This could balance the label distribution between background and T.I. regions, thereby improving the suitability of the segmented images for subsequent analyses. Ultimately, these initiatives aim to overcome the challenges identified in our exploration of ground truth segmentations.


