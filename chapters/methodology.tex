The methodology is essential for the success of our segmentation task. In this chapter, we will discuss the key components of our methodology, including dataset preprocessing, baseline implementation, weak label generation, choice of the training settings, training pipeline and model inference. As the MRI data with different modalities presents varying features during model training, our methodology will be applied separately to both Axial and Coronal MRI and further investigate the cross-modality segmentation results. 

\section{Dataset Preprocessing}

Our preliminary data analysis unveiled that non-background voxels comprise only a small fraction of the total MRI volume. As a result, they may not provide substantial information to facilitate effective learning for the segmentation model. To address this limitation, we harness the provided centreline coordinates of the terminal ileum to define a bounding box encompassing the T.I. region. Subsequently, the bounding box information will be encoded to prompts for refining the weak label generation with MedSAM. Details for this step will be covered in the model training part later.

Furthermore, we employ traditional image processing methods to streamline the data further. These include resizing and image normalisation, which abate the complexity of the data while maintaining consistency within the dataset.

Before commencing training, we must format the dataset following nnU-Net's requirements \cite{isensee2021nnu}. After a successful conversion, nnU-Net undertakes to preprocess by adjusting images to various dimensions and resolutions and optimising them for ensemble model training. Simultaneously, it generates a unique \textbf{`dataset fingerprint'}, which proves invaluable during subsequent hyperparameter tuning. With these prerequisites satisfied, the preprocessed dataset stands ready for proxy learning.

This rigorous and systematic approach to data preprocessing ensures our model is equipped with well-structured data, which is critical in enhancing its learning efficiency. Consequently, this results in improved accuracy of the generated segmentations, thus underlining the importance of adequate dataset preprocessing in clinical imaging analysis.

\section{Baseline Implementation}

Building upon precedent research, we've formulated a baseline model that adheres to a systematic and robust methodology. This model lays the groundwork for further enhancement and modifications later in our project:

Our first step involves the application of 3D Simple Linear Iterative Clustering (SLIC) superpixel segmentation on the cropped Region of Interest (ROI), where the ROI is defined through centerline coordinates. This task generates a weak mask that becomes instrumental in our proxy learning stage. To ensure the successful execution of this step, we identify the superpixels situated on the centreline of the terminal ileum, leveraging provided centreline coordinates. Subsequently, the segmented superpixels are relabelled to binary labels, as it realigns output with our project objectives.

Upon generation of the initial segmentation, we confront the potential issue of a hole or tube-like structures within the segmented supervoxels. These irregularities can disrupt the continuity of the terminal ileum representation. Therefore, our process incorporates a voting iterative binary hole-filling algorithm and morphological hole-closing to create a more refined segmentation.

With the generation of weak masks accomplished, we train our proxy model. Furthermore, a second training iteration is conducted, deploying the proxy model on fully-annotated data to obtain the final segmentation model. The following sections will discuss additional aspects of the training process and the ensembling strategy employed to derive the final model.

\section{Model Training}
As we step into our process, our attention is directed towards the pivotal stage of model training. This stage is characterised by an enriched proxy learning strategy, where we replace the initial unsupervised 3D SLIC-based weak mask generation with a more sophisticated tool: MedSAM. The training and evaluation phases are also designed to optimise model performance, incorporating cross-validation and ensemble learning methods. Below, we elaborate on these critical aspects:

\begin{itemize}
\item \textbf{MedSAM Application}:
In this subsection, we discuss in detail how we leverage MedSAM for generating high-quality weak masks, adapting our MRI dataset to MedSAM via fine-tuning and elucidating its superior capabilities in medical image segmentation and its specific role in enhancing our model learning trajectory.

\item \textbf{Dataset Partitioning}:
Here, we explain how we divide our dataset into training and testing sets using an 80/20 ratio. We provide a rationale for this split and discuss how it ensures a robust and unbiased evaluation of our model performance.

\item \textbf{Training Pipeline}:
Our multifaceted training pipeline starts with a two-phased approach. Initially, we train a proxy model on weak masks generated by MedSAM. This is followed by further training on 80\%  of the fully annotated data, maximizing learning from weak and fully annotated data sets.

To ensure model robustness, a 5-fold cross-validation method is integrated during training to mitigate overfitting risks. Furthermore, our pipeline supports different configurations, which include 2D and 3D variants of data, loss function adjustments, and variations in training optimisers and learning rates.

The final aspect of our pipeline is ensemble learning, utilized to aggregate models across different folds and configurations, thereby generating our target model. The collective operation of these facets is meticulously designed to bolster our model performance in diverse scenarios.

\item \textbf{Inference}:
In this final stage, we delve into utilising the trained model for making inferences. We outline the procedural aspects of applying the model to new and unseen data, thus generating predictive outcomes that form the basis for the subsequent comprehensive evaluation of our model performance. The evaluation process, with detailed quantitative analysis, will be discussed extensively in the following chapter.
\end{itemize}

Through this comprehensively planned model training phase, we aim to build a proficient terminal ileum segmentation model that learns effectively from our data and showcases robust performance across varying scenarios.

\subsection{Refining Weak Masks with MedSAM}

The enhancement of our weak masks is a crucial cornerstone of our strategy. To achieve this, we utilize MedSAM - a modified version of the SegmentAnything Model specifically designed for medical images. Its bespoke design facilitates the creation of finely-tuned and detailed weak masks, setting a solid foundation for superior segmentation outcomes.

Our journey begins by introducing our preprocessed images to the MedSAM model. Traditional methodologies often fall short when compared with this advanced tool. MedSAM's core strength lies in its innovative AI algorithms identifying unique attributes embedded within medical images. This capability enables the production of granular segmentations far surpassing the precision attainable through conventional models. A fine-tuning process is executed to reconfigure the MedSAM model for compatibility with our dataset and ensure the generation of high-quality, refined masks. This involves training the pre-trained MedSAM model on fully annotated ground truth segmentations.

Once the fine-tuning phase wraps up, we leverage SAM to generate weak masks on images devoid of full annotations. An integral part of our weak label generation includes bounding boxes focusing on the T.I. region, as defined by centerline coordinates. As substantiated by Huang et al., 2023 \cite{huang2023segment}, such augmentation amplifies the quality of the resulting weak mask. For those images that lack provided centerline data, we opt to exclude them from our dataset. This ensures a coherent and reliable pool of data driving our methodology forward.

Incorporating MedSAM into our workflow signifies a pivotal stride towards achieving phenomenal segmentation outcomes. The emergence of refined masks unveils new dimensions of learning for our model, cementing a robust foundation for subsequent breakthroughs in terminal ileum segmentation.

\subsection{Executing Dataset Partitioning}

Partitioning our dataset into training and testing sets is crucial in our model development. This strategy helps assess the model performance on new, unseen data, establishing a reliable measure of its predictive capabilities.

Our approach involves a partitioning of the dataset into an 80/20 ratio. Before this division, we shuffle the dataset to randomise the order of samples, ensuring an unbiased representation in both the training and testing sets. Subsequently, the first 80\% of these randomized samples form our training set, while the remaining 20\% are reserved for testing.

The reason for using such a ratio is that it is widely revered in machine learning and serves to strike a balance between providing sufficient data to nourish the model's learning journey and withholding a substantial portion for an authentic evaluation of its performance. This balance becomes particularly crucial when navigating through scenarios characterised by a limited dataset, such as ours.

This cautious allocation of data equips us with a model that's not only well-trained but also rigorously tested for its predictive performance, strengthening our confidence in its segmentation ability.

\subsection{Establishing the Training Pipeline}
This subsection outlines on principal components of our training pipeline, including the deliberate choices and reasonings behind formulating the loss function, selecting the optimiser and determining the learning rate. A two-phased training approach with the key components is proposed to ensure the model's capability to learn efficiently, navigate an optimised prediction path, and deliver robust T.I. segmentation.
\subsubsection*{Loss Function Formulation}
Guiding the learning trajectory of our model during training is an elegantly designed loss function. Particularly, we harness nnU-Net's unique blend of cross-entropy and Dice losses \cite{drozdzal2016importance}, expressed as:

\[
\mathcal{L} = \mathcal{L}_{CE} + \mathcal{L}_{Dice}
\]

For our specific use case, this composite loss function manifests into a binary form, simplifying to a binary cross-entropy loss defined as:

\[
\mathcal{L}_{CE} = \sum_{i}^{N}\left(y_{i}\log o_{i} + (1 - y_{i}) \log (1 - o_{i})\right)
\]

In this equation, \(y_{i} \in \{ 0, 1 \}\) symbolizes the ground truth label for the \(i\)-th instance, while \(o_{i} \in \left[0, 1\right]\) represents the corresponding sigmoid probability for the given label \(y_{i}\).

Complementing this, the Dice loss is articulated as:

\[
\mathcal{L}_{Dice} = -\frac{2\sum_{i}^{N}o_{i}y_{i}}{\sum_{i=1}^{N}\left(o_{i} + y_{i}\right)}
\]

Here, \(N\) encapsulates the total number of pixels in the training batch.

Carefully orchestrating these two components within the loss function empowers our model to traverse the complex learning landscape effectively, optimising its performance in terminal ileum segmentation.

\subsubsection*{Optimiser Selection}

In constructing an effective training strategy for our model, we deploy Stochastic Gradient Descent (SGD) as the principal optimizer, supplemented by a Nesterov momentum set to 0.99.

This process can be imagined as an expert guide traversing complex terrain, searching for the lowest point or valley - an analogy for the optimal solution that minimizes error. In this context, SGD serves as our skilled explorer, persistently heading towards the steepest downward gradient in pursuit of the valley.

However, as any experienced guide attests, the steepest descent does not necessarily lead to the lowest valley due to potential undulations and terrain variations.

Addressing this challenge is the role of the Nesterov momentum. It equips our guide with a metaphorical telescope, allowing for a foresighted view of the landscape along the current path before finalising the next step. This foresight permits more informed decisions considering the overall landscape rather than the immediate surroundings.

This stands in contrast to the classical momentum method, which can be considered a hiker relying solely on their current position and pace to determine their next step without any foresight or scouting tools. An intuitive illustration of the difference between the effect of classical and Nesterov Momentum is shown in \autoref{fig:nesterov} \cite{lectures16:online}.

\begin{figure}[htp]
    \centering
    \includegraphics[width=0.8\textwidth]{./figures/nesterov.png}
    \caption{Nesterov momentum}
    \label{fig:nesterov}
\end{figure}

Employing Nesterov momentum with our SGD optimizer enables a quicker and more precise journey to the `valley' or optimal solution, ensuring our model learns effectively and efficiently from our data.

To illustrate this concept more clearly, \autoref{alg:sgd-nesterov} outlines this process in detail:
\begin{algorithm}[htp]
\caption{SGD enhanced with Nesterov Momentum}\label{alg:sgd-nesterov}
\begin{algorithmic}[1]
\Require Initial learning rate \(\eta\), momentum \(\beta\)
\Require Initial parameter \(\mathbf{\theta_{0}}\), initial velocity vector \(\mathbf{v_{0}}\)
\While{not reaching stopping criteria}
\item Draw one batch with \(m\) samples \({ x^{(1)}, \ldots , x^{(m)} }\) with corresponding \(y^{(i)}\)
\item Temporarily update the parameter: \(\tilde{\mathbf{\theta_{t}}} \leftarrow \mathbf{\theta_{t}} - \beta \mathbf{v_{t}}\)
\item Compute the gradient at the adjusted point:
\[
\mathbf{g_{t}} \leftarrow \nabla_{\tilde{\mathbf{\theta_{t}}}}\sum_{i}\mathcal{L}(f(x^{(i)}; \tilde{\mathbf{\theta_{t}}}), y^{(i)})
\]
\item Refresh the velocity: \(\mathbf{v_{t+1}} \leftarrow \beta\mathbf{v_{t}} + \eta \mathbf{g_{t}}\)
\item Update the parameter: \(\mathbf{\theta_{t + 1}} \leftarrow \mathbf{\theta_{t}} - \mathbf{v_{t + 1}}\)
\EndWhile
\end{algorithmic}
\end{algorithm}

In more technical terms, the Nesterov momentum method facilitates accelerated learning relative to traditional methods. This acceleration is of crucial strategic value, as it leads to significant time and resource efficiencies, thereby boosting the progression of our model into a highly competent predictive tool in a shorter span.

\subsubsection*{Learning Rate Determination}
The learning rate is crucial in optimising our model, which guides the magnitude of updates made to learnable parameters during each iteration. We have set an initial learning rate of 0.01 to balance by ensuring that our model learns steadily without missing out or skipping over essential information.

However, the critical characteristic of our approach is that our learning rate is not constant — it gradually decreases as training progresses. This strategy, known as learning rate annealing or decay, is essential in optimising our model performance.

As the training advances, model parameters tend to converge towards an optimal solution. Here, making significant adjustments is not ideal because it might cause the parameters to oscillate around (or overshoot) the desired optimum. To address this issue, we decrease the learning rate across iterations, encouraging smaller steps when we are presumably closer to the optimum and enhancing the model performance over time.

We employ a learning rate scheduler that uses a polynomial function determined by the total number of training iterations. The decaying learning rate, \( \eta_{t} \), is computed by:
\[
\eta_{t} = \eta_{0}(1 - \frac{t}{N})^{0.9}
\]

Here, \( \eta_{0} = 0.01 \) is the initial learning rate, \( t \) stands for the current iteration count, and \( N \) represents the total iterations.

By carefully managing the learning rate via this method, we ensure that our model can adapt effectively across training steps, continually refining its performance with each iteration.

\subsubsection*{Two-phased training}
Our training pipeline adopts a two-phased strategy, heavily leveraging the loss function, optimizer, and learning rate detailed prior.

In the first phase, the proxy model is trained using the generated weak mask over 200 epochs, reserving 20\% of the training data for validation. This process utilises an SGD optimiser with Nesterov momentum and an initial learning rate of 0.01 with our proposed strategy of learning rate decay. After completing this preliminary phase, nnU-Net collates the validation results and combines them with pre-extracted dataset fingerprints to finalise the optimal hyperparameter combination for the proxy model.

The subsequent phase fine-tunes the proxy model on fully annotated images across 50 epochs. A 5-fold cross-validation along with the same optimisation settings as the previous phase ensures a compelling learning trajectory. Notably, only 80\% of the fully annotated data feeds into this stage, preserving 20\% of the data and their corresponding ground truth segmentations for the ultimate evaluation.

This two-phase strategy remains adaptable to varied data configurations, accommodating differences in image dimensions and resolutions while seamlessly integrating with 2D and 3D full-resolution data. We ensure each model concludes its training and strategically combine predictions on unseen gold standard data to determine the final output.

The selection of the final model is guided by cross-validation performance. Depending on the outcomes, the final model could either be a single best-performing model or an ensemble of models trained across different folds. Our objective remains consistent throughout this meticulously planned two-phase training: To create a proficient model capable of delivering exceptional terminal ileum segmentation.

\subsection{Executing the Inference Process}

Once the optimal model is determined, we can introduce unseen, pre-processed data into the nnU-Net framework. This enables us to generate straightforward predictions. However, it is important to highlight several points that characterise this predictive phase.

In alignment with nnU-Net's patch-based training procedure, inference also adopts a patch-based methodology. Each image is divided into smaller sub-images or `patches', and it is upon these patches that our model bases its predictions. Nevertheless, another worth-mentioning point is that the model precision tends to decrease towards the edge of these patches. Consequently, when generating predictions, the model attributes greater significance to the voxels near the patch's centre than their edge-located counterparts. This strategy ensures a notably higher prediction quality upon aggregating the predictions across all patches.

Upon the completion of the model prediction, a common practice is to apply post-processing to the generated predictions. Post-processing is often employed based on connected components to enhance image segmentation, such as organs. This approach typically centres on disregarding smaller, potentially insignificant elements and emphasising the most expansive interconnected region to mitigate the probability of false positives.

Exemplifying this philosophy, nnU-Net systematically utilises the implications of omitting these smaller entities on the model performance, employing cross-validation results as a reference metric. Initially, each foreground class is treated as a singular entity. If the constraint to the most significant region increases the average foreground Dice coefficient without diminishing any class-specific coefficients, this method emerges as the preliminary post-processing step. After the outcome of this stage, nnU-Net determines whether the same way requires the application to individual classes.

This sophisticated, multi-faceted approach to inference empowers us to maximise the accuracy, reliability, and clinical relevance of our terminal ileum segmentation predictions.