Crohn's Disease \cite{baumgart2012crohn,crohnsNHS} is known as one of the two major types of Inflammatory Bowel Disease (IBD) \cite{IBDCDC}. Patients with IBD will be suffered from serious and chronic inflammation of the gastrointestinal (GI) tract. \medskip

\noindent The main concern raised from such symptoms is that the chronic and long-term inflammation can lead to damage in GI tract, along with further possible complications, including bowel cancer. A recent research from the University of Nottingham \cite{UoNResearch} states that more than half a million people in in UK are living with Crohn's Disease and Ulcerative Colitus, another main type of IBD. \medskip

\noindent However, unlike Ulcerative Colitus, which only affects colon and rectum, Crohn's disease can develop leisions in any part of the GI tract, from the oesophagus to the anus. This poses two main problems. Since inflammation can occur in any part of the GI tract, the patient's entire digestive process can be affected. Also, due to the diversity of affected locations, the symptom of this disease will vary patient to patient, including abdominal pain, diarrehoea, fatigue, and also weight loss. \medskip

\noindent Although the medical community has invested a great deal of research into Crohn's disease \cite{hoarau2016bacteriome,feuerstein2021aga}, the exact cause of Crohn's disease has not been identified, therefore Crohn's disease is not curable until today. \medskip

\noindent Fortunately enough, early diagnosis coupled with appropriate treatment from clinical professionals can greatly reduce the patient's suffering and improve the quality of their lives. Typical diagnosis can be done through various approaches such as enteroclysis, edoscopy, colonoscopy, and radiographic diagnosis including barium contrast X-rays, Computed Tomography (CT) and Magnetic Resonance Imaging (MRI) etc. The radiographic diagnosis is a more preferable among the all since it provides a non-invasive approach for the suspected intra-abdominal abscesses examination. \medskip

\noindent The examiners will look particularly into the terminal ileum and right colon, where the areas are most involed in in Crohon's diease, for symptoms such as bowel wall thickening, small bowel strictures and mural hyperenhancement \cite{bruining2018consensus} to confirm the existence of Crohn's disease. Nevertheless, human experts still need to go through the 2D MRI scans slice by slice, which is time-consuming and inefficient to diagnose one patient. \medskip

\noindent The prevalance of Machine Learning and Deep Learning, especially convolutional neural networks (CNN), provides us a possiblity to learn the features from input imaging data automatically, and help the human expert for diagnosis. Given that the terminal ileum is essential in Crohn's Disease diagnosis, in 2019, Holland et al. \cite{holland2019automatic} proposed a residual network focusing on terminal ileum to automatically detect Crohn's Disease from MRI scans. They concluded that the performance of the framework is dependent on the level of localisation in preprocessing, and recommend that terminal ileal ground-truth segmentations also be collected for localising the terminal ileum and enhancing the performance of automated detection. \medskip

\noindent A follow-up study by Xu et al. \cite{KeXu2021} in 2021 developed a deep learning tool using a nnU-Net architecuture to automatically locate critical areas (particularly the end of the ileum) where radiologists typically perform examinations to make a diagnosis. This study resolves the high dependence on the level of localisation in preprocessing in \cite{holland2019automatic}, and the developed target model is able to localise and decipt the key regions for Crohn's disease detection. \medskip

\noindent Inspired by studies of \cite{holland2019automatic,KeXu2021}, the objective of this individual project is to imporve the segmentation precision by refining the data pre-processing pipeline including image co-registration, and introduce a complete setup of nnU-Net with an additional 2D U-Net or 3D cascade. Further a more advanced nnU-Net based architecture will be used in the training stage for getting higher and more robust performance. Using image co-registration will make the target model learn a more comprehesnsive object representation. The complete setup will potentially boosts the performance of the segmentation model, and further maximise the capability to the segmentation task. \medskip

\noindent Both imporvements will lead to a more precise key-region localisation and benefit both patient and healthcare provider with a more efficient diagnosis and more effective treatment.
