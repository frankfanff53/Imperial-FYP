\section{Crohn's Disease}

Crohn's Disease \cite{baumgart2012crohn,crohnsNHS} is known as one of the two main types of Inflammatory Bowel Disease (IBD) \cite{IBDCDC}. Patients with IBD will suffer from severe and chronic inflammation of the gastrointestinal (GI) tract. \medskip

\noindent The main concern raised from such symptoms is that chronic and long-term inflammation can lead to damage in the GI tract, along with further possible complications, including bowel cancer. Recent research from the University of Nottingham \cite{UoNResearch} states that more than half a million people in the UK live with Crohn's Disease and Ulcerative colitis, another primary type of IBD. \medskip

\noindent However, unlike Ulcerative Colitis, which only affects the colon and rectum, Crohn's disease can develop lesions in any part of the GI tract, from the oesophagus to the anus. This fact poses two main problems. Since inflammation can occur in any part of the GI tract, the patient's digestive process can be affected. Also, due to the diversity of affected locations, the symptom of this disease will vary from patient to patient, including abdominal pain, diarrhoea, fatigue, and weight loss. 
\section{Clinical Challenges}

\noindent Although the medical community has invested much research into Crohn's disease \cite{hoarau2016bacteriome,feuerstein2021aga}, our clinical experts have not identified the exact cause of Crohn's disease yet. Therefore Crohn's disease is not curable today. \medskip

\noindent Fortunately enough, early diagnosis coupled with appropriate treatment from clinical professionals can significantly reduce the patient's suffering and improve the quality of their lives. Clinicians typically make a specific diagnosis through various approaches such as enteroclysis, endoscopy, colonoscopy, and radiographic diagnosis, including barium contrast X-rays, Computed Tomography (CT) and Magnetic Resonance Imaging (MRI). The MRI diagnosis has become popular since it is a non-invasive approach and produces more detailed images than its alternatives, such as computerised tomography (CT). \medskip

\noindent The examiners will look particularly into the terminal ileum, and right colon, where the areas are most involved in Crohn's disease, for symptoms such as bowel wall thickening, small bowel strictures, and mural hyperenhancement \cite{bruining2018consensus} to confirm the existence of Crohn's disease. Nevertheless, human experts still need to go through the MRI scans slice by slice, which is time-consuming and inefficient to diagnose one patient. 
\section{Motivation}
The prevalence of Machine Learning and Deep Learning, especially convolutional neural networks (CNN), allows us to automatically learn the features from input imaging data and help the human expert with diagnosis. Given that the terminal ileum is essential in Crohn's Disease diagnosis, in 2019, Holland et al. \cite{holland2019automatic} proposed a residual network focusing on the terminal ileum to detect Crohn's Disease from MRI scans automatically. They concluded that the framework's performance depends on the level of localisation in preprocessing. Consequently, they recommend that terminal ileal ground-truth segmentations also be collected for localising the terminal ileum and enhancing the performance of automated detection. \medskip

\noindent A follow-up study by Abidi et al. \cite{Ali2022} in 2022 developed an enhanced deep learning tool using a nnU-Net architecture to automatically locate critical areas (particularly terminal ileum) where radiologists typically perform examinations to make a diagnosis. This study resolves the high dependence on the level of localisation in preprocessing in \cite{holland2019automatic}, and the model significantly outperforms the results of the prior work done by Xu et al. \cite{KeXu2021} in 2021 for terminal ileum segmentation. Furthermore, \cite{Ali2022} establishes a strong foundation for a multi-class terminal ileum segmentation algorithm. \medskip

\noindent Inspired by studies of \cite{holland2019automatic, KeXu2021, Ali2022}, this individual project aims to develop a multi-class segmentation model to differentiate normal and abnormal terminal ileum regions, and the clinicians will be significantly aided in the diagnosis of Crohn's disease with the success of this project.

\section{Machine learning Challeges}
The performance of a deep learning model is strongly related to the training data's quality, quantity and diversity. Unfortunately, our biggest challenge in this project is the scarcity of training data. Our training dataset is relatively small (only 233 patient cases available) compared with other industry-leading deep learning systems. Another fact we need to face is that developing gold-standard labels or point-wise centerlines for patient data requires manual segmentations by clinical experts, which is also a complex, time-consuming and inefficient process. Therefore, the acquisition of patient data with great quality and quantity is indeed a challenging process. \medskip

\noindent To alleviate these problems, we introduce a proxy task with weak supervision before the target training task to get comparable results with learning a large number of data with weak labels that are less spatially informative. However, from prior work \cite{Ali2022}, it is suggested that training from scratch using the nnU-Net framework in the proxy task is inefficient since it takes a long time to converge and exhibits unstable performance. Applying transfer learning to proxy learning \cite{jang2021effectiveness} or adding a related pre-training task may resolve this situation.

\section{Objectives}
The primary objectives of our work include the following:
\begin{enumerate}
    \item Replicating the results of the prior work \cite{Ali2022} for segmenting the terminal ileum from MRI scans.
    \item Developing a multi-class segmentation model of the terminal ileum based on the prior work and employing the trained model as our baseline model. 
    \item Developing a sophisticated, related pre-training task for the multi-class terminal ileum segmentation that leverages the industrial leading transfer learning techniques to imporve the performance of the proxy model and associated stability.
    \item Improving the results of the baseline segmentation model and developing a user-friendly GUI as an extension to this project.
\end{enumerate}