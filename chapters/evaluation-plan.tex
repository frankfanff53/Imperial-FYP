Ensuring the effectiveness of our approach in developing an advanced segmentation model for the terminal ileum necessitates formulating a rigorous evaluation plan. This plan outlines systematic stages and metrics to assess our system's performance and alignment with our project objectives:

\begin{enumerate}
    \item \textbf{Baseline Model Evaluation}: The initial step involves establishing and evaluating the performance of the nnU-Net-based segmentation model. We conduct baseline tests using annotated data to measure its initial performance, specifically focusing on Dice Similarity Coefficient (DSC) as an indicative metric. These results will serve as our benchmark against which to evaluate further improvements.
    \item \textbf{Generation and Inspection of Coarse-grained Weak Masks}: Leveraging the Simple Linear Iterative Clustering (SLIC) method, we generate coarse-grained weak masks for data lacking ground-truth annotations. We will evaluate the produced masks, both qualitatively and quantitatively, to gauge the quality and adequacy of these initial segmentation outputs.
    \item \textbf{Proxy Model Training and Evaluation}: Following the production of coarse-grained weak masks, we employ them to train the proxy model. We evaluate its performance based on a series of parameters: training stability, learning rate, convergence speed, and DSC score, among others. The results obtained provide insight into the efficacy of using SLIC-created masks in training models.
    \item \textbf{Integration and Evaluation of SegmentAnything Model}: To refine the generated weak masks, we integrate the SegmentAnything Model from Meta AI into our pipeline, with additional information of the ROI, including centerline coordinates and bounding boxes provided. We will assess the quality of the finer-grained weak masks produced by SAM through visual inspection and statistical measures, comparing them against the coarse-grained masks created by SLIC.
    \item \textbf{Target Model Training and Evaluation}: Utilizing the refined weak masks and fully annotated data, we develop the target segmentation model. The performance of this model is evaluated based on important parameters such as DSC, training efficiency, and generalization gap.
    \item \textbf{Overall Performance Evaluation}: In the final stage, we perform a comprehensive analysis to compare our derived models with our original objectives and benchmark model. Ultimately, we aim to identify the most significant performance enhancements, evaluate the value of our model-refinement process, and highlight possible directions for future research.
\end{enumerate}

By implementing this robust evaluation plan, we strive to validate the success of our project through meticulous testing, thereby assuring the effectiveness and accuracy of our terminal ileum segmentation model.